% Options for packages loaded elsewhere
\PassOptionsToPackage{unicode}{hyperref}
\PassOptionsToPackage{hyphens}{url}
%
\documentclass[
]{book}
\usepackage{lmodern}
\usepackage{amsmath}
\usepackage{ifxetex,ifluatex}
\ifnum 0\ifxetex 1\fi\ifluatex 1\fi=0 % if pdftex
  \usepackage[T1]{fontenc}
  \usepackage[utf8]{inputenc}
  \usepackage{textcomp} % provide euro and other symbols
  \usepackage{amssymb}
\else % if luatex or xetex
  \usepackage{unicode-math}
  \defaultfontfeatures{Scale=MatchLowercase}
  \defaultfontfeatures[\rmfamily]{Ligatures=TeX,Scale=1}
\fi
% Use upquote if available, for straight quotes in verbatim environments
\IfFileExists{upquote.sty}{\usepackage{upquote}}{}
\IfFileExists{microtype.sty}{% use microtype if available
  \usepackage[]{microtype}
  \UseMicrotypeSet[protrusion]{basicmath} % disable protrusion for tt fonts
}{}
\makeatletter
\@ifundefined{KOMAClassName}{% if non-KOMA class
  \IfFileExists{parskip.sty}{%
    \usepackage{parskip}
  }{% else
    \setlength{\parindent}{0pt}
    \setlength{\parskip}{6pt plus 2pt minus 1pt}}
}{% if KOMA class
  \KOMAoptions{parskip=half}}
\makeatother
\usepackage{xcolor}
\IfFileExists{xurl.sty}{\usepackage{xurl}}{} % add URL line breaks if available
\IfFileExists{bookmark.sty}{\usepackage{bookmark}}{\usepackage{hyperref}}
\hypersetup{
  pdftitle={Traitement vidéo},
  pdfauthor={Guillaume Arseneault},
  hidelinks,
  pdfcreator={LaTeX via pandoc}}
\urlstyle{same} % disable monospaced font for URLs
\usepackage{longtable,booktabs}
\usepackage{calc} % for calculating minipage widths
% Correct order of tables after \paragraph or \subparagraph
\usepackage{etoolbox}
\makeatletter
\patchcmd\longtable{\par}{\if@noskipsec\mbox{}\fi\par}{}{}
\makeatother
% Allow footnotes in longtable head/foot
\IfFileExists{footnotehyper.sty}{\usepackage{footnotehyper}}{\usepackage{footnote}}
\makesavenoteenv{longtable}
\usepackage{graphicx}
\makeatletter
\def\maxwidth{\ifdim\Gin@nat@width>\linewidth\linewidth\else\Gin@nat@width\fi}
\def\maxheight{\ifdim\Gin@nat@height>\textheight\textheight\else\Gin@nat@height\fi}
\makeatother
% Scale images if necessary, so that they will not overflow the page
% margins by default, and it is still possible to overwrite the defaults
% using explicit options in \includegraphics[width, height, ...]{}
\setkeys{Gin}{width=\maxwidth,height=\maxheight,keepaspectratio}
% Set default figure placement to htbp
\makeatletter
\def\fps@figure{htbp}
\makeatother
\setlength{\emergencystretch}{3em} % prevent overfull lines
\providecommand{\tightlist}{%
  \setlength{\itemsep}{0pt}\setlength{\parskip}{0pt}}
\setcounter{secnumdepth}{5}
\usepackage{booktabs}
\usepackage{amsthm}
\makeatletter
\def\thm@space@setup{%
  \thm@preskip=8pt plus 2pt minus 4pt
  \thm@postskip=\thm@preskip
}
\makeatother
\ifluatex
  \usepackage{selnolig}  % disable illegal ligatures
\fi
\usepackage[]{natbib}
\bibliographystyle{apalike}

\title{Traitement vidéo}
\author{Guillaume Arseneault}
\date{2021-01-11}

\begin{document}
\maketitle

{
\setcounter{tocdepth}{1}
\tableofcontents
}
\listoftables
\listoffigures
\hypertarget{pruxe9face}{%
\chapter{Préface}\label{pruxe9face}}

Ce livre se produit via \textbf{bookdown} \citep{rmarkdown2018}, R Markdown et \textbf{knitr} \citep{xie2015}

\hypertarget{traitement-viduxe9o-582-543-mo}{%
\chapter{Traitement vidéo (582-543-MO)}\label{traitement-viduxe9o-582-543-mo}}

\hypertarget{description-du-cours}{%
\section{Description du cours}\label{description-du-cours}}

\begin{itemize}
\tightlist
\item
  Techniques D'INTÉGRATION MULTIMÉDIA
\item
  Département des techniques d'intégration multimédia
\item
  582.A1
\item
  Pondération : 1-2-2
\item
  Unités: 1,66
\item
  Heures-contact : 45
\item
  Session : 4
\end{itemize}

Ce cours permet à l'étudiante ou l'étudiant d'enregistrer, de modifier et de traiter des images en temps réel.
L'étudiant sera appelé à appliquer des effets visuels aux images vidéo et à adapter les images en fonction de l'intégration.

\hypertarget{objectifs}{%
\section{Objectifs}\label{objectifs}}

\hypertarget{objectif-intuxe9grateur-et-ministuxe9riel}{%
\subsection{Objectif intégrateur et ministériel}\label{objectif-intuxe9grateur-et-ministuxe9riel}}

\begin{itemize}
\tightlist
\item
  015J Traiter les images en mouvement
\end{itemize}

\hypertarget{objectifs-dapprentissage}{%
\subsection{Objectifs d'apprentissage}\label{objectifs-dapprentissage}}

\begin{itemize}
\tightlist
\item
  Adapter des images en mouvement (Importance relative\,: 40\% )
\item
  Programmer des effets spéciaux et l'interactivité (Importance relative\,: 40\% )
\item
  Intégrer des images en mouvement à une production interactive (Importance relative\,: 20\% )
\end{itemize}

\hypertarget{attitudes-professionnelles}{%
\subsection{Attitudes professionnelles}\label{attitudes-professionnelles}}

\begin{itemize}
\tightlist
\item
  Créativité
\item
  Sens esthétique
\item
  Adaptation
\end{itemize}

\hypertarget{habiletuxe9s-transdisciplinaires}{%
\subsection{Habiletés transdisciplinaires}\label{habiletuxe9s-transdisciplinaires}}

Profil TIC~: les étudiantes et étudiants auront à exploiter les TIC de manière efficace et responsable. Ils auront à rechercher, à traiter et à présenter de l'information.

\hypertarget{pruxe9alables}{%
\section{Préalables}\label{pruxe9alables}}

\hypertarget{pruxe9alable-absolu-au-pruxe9sent-cours}{%
\subsection{Préalable absolu au présent cours~:}\label{pruxe9alable-absolu-au-pruxe9sent-cours}}

\begin{itemize}
\tightlist
\item
  582 413 MO Montage vidéo
\end{itemize}

\hypertarget{pruxe9alable-absolu-aux-cours-suivants}{%
\subsection{Préalable absolu aux cours suivants~:}\label{pruxe9alable-absolu-aux-cours-suivants}}

\begin{itemize}
\tightlist
\item
  582~513 MO Conception de projet multimédia
\item
  582 66B MO Expérience multimédia interactive
\item
  582 66G MO Production Web en entreprise
\end{itemize}

\hypertarget{contexte-particulier-dapprentissage}{%
\section{Contexte particulier d'apprentissage}\label{contexte-particulier-dapprentissage}}

\begin{itemize}
\tightlist
\item
  En laboratoire et studio.
\end{itemize}

\hypertarget{fiche-technique}{%
\subsection{Fiche technique}\label{fiche-technique}}

\begin{itemize}
\tightlist
\item
  Ordinateurs, projecteurs à haute luminosité ou télévision, haut-parleurs professionnels, casque audio, et tout le matériel disponible pour TIM
\item
  Logiciels de montage vidéo et traitemet vidéo en temps réel

  \begin{itemize}
  \tightlist
  \item
    Open broadcast studio
  \item
    Unity
  \item
    Pure Data
  \item
    Resolve
  \item
    Reaper
  \item
    ffmpeg
  \item
    Open stage control
  \end{itemize}
\item
  Languages et protocoles

  \begin{itemize}
  \tightlist
  \item
    Programmation nodale
  \item
    Javascript
  \item
    Open sound control (OSC)
  \item
    Réseautique (addressage ip)
  \item
    Midi
  \item
    NDI
  \item
    Websocket
  \end{itemize}
\end{itemize}

Technicienne ou technicien en travaux pratiques

\hypertarget{contenus-essentiels}{%
\section{Contenus essentiels}\label{contenus-essentiels}}

\hypertarget{survol-historique}{%
\subsection{Survol historique}\label{survol-historique}}

\begin{itemize}
\tightlist
\item
  Évolution historique du traitement vidéo dans les différentes formes d'art

  \begin{itemize}
  \tightlist
  \item
    Performance
  \item
    Installation
  \item
    Évolution des technologies associées
  \end{itemize}
\item
  Langages et moyens expressifs de l'image en mouvement
\end{itemize}

\hypertarget{fondements-technique}{%
\subsection{Fondements technique}\label{fondements-technique}}

\begin{itemize}
\tightlist
\item
  Formats de fichiers
\item
  Encodage des vidéos\\
\item
  Captation vidéo en temps réel
\item
  Logiciels de traitement vidéo en temps réel et d'interactivité
\item
  Logiciels de programmation nodale
\item
  Notions de traitement vidéo

  \begin{itemize}
  \tightlist
  \item
    pixels,
  \item
    couleurs,
  \item
    texture,
  \item
    matrice,
  \item
    mémoire tampon
  \item
    alpha channel
  \item
    rendu OpenGL
  \end{itemize}
\end{itemize}

\hypertarget{traitement-de-limages-en-mouvement}{%
\subsection{Traitement de l'images en mouvement}\label{traitement-de-limages-en-mouvement}}

\begin{itemize}
\tightlist
\item
  Usage de capture vidéo en temps réel\\
\item
  Effets visuels et filtres applicables en temps réel sur des matériaux visuels\\
\item
  Traitement visuel en temps réel à l'aide d'effets et de logiciels de programmation multimédia et nodale
\item
  Flot de données entre les objets du logiciel
\item
  Exploitation des fonctions des logiciels de traitement vidéo en temps réel
\item
  Utilisation de nuanceurs (shaders)
\end{itemize}

\hypertarget{programmation-deffets-visuels}{%
\subsection{Programmation d'effets visuels}\label{programmation-deffets-visuels}}

\begin{itemize}
\tightlist
\item
  Programmation de compositions visuelles génératives
\item
  Réalisation d'un échantillonneur/mixeur visuel
\item
  Programmation pour contrôler la lecture vidéo,

  \begin{itemize}
  \tightlist
  \item
    montage temps réel
  \item
    niveau des couleurs
  \item
    alpha channel\\
  \end{itemize}
\item
  Programmation nodale pour créer des effets en temps réel

  \begin{itemize}
  \tightlist
  \item
    position
  \item
    rotation
  \item
    dimensions
  \item
    mixage d'images
  \item
    incrustation
  \item
    distorsion
  \item
    délais
  \item
    rétroaction (feedback)
  \item
    modification de couleurs
  \item
    chromakey
  \item
    lumière
  \item
    fumée
  \item
    texture
  \end{itemize}
\item
  Nuanceurs (shaders)\,: vertex, pixel et géométrie
\end{itemize}

\hypertarget{image-en-mouvement-et-interactivituxe9}{%
\subsection{Image en mouvement et interactivité}\label{image-en-mouvement-et-interactivituxe9}}

\begin{itemize}
\tightlist
\item
  Intégration des composantes dans une production interactive
\item
  Configuration logicielle et matérielle d'une production interactive\\
\item
  Conceptualisation et scénarisation d'un projet visuel interactif\\
\item
  Captation de mouvement et de présence
\item
  Programmation de la captation de mouvement et de présence
\item
  Utilisation d'interfaces de contrôle interactives
\item
  Utilisation d'OSC, MIDI, DMX ou ArtNet pour interagir avec d'autre logiciels et interfaces de contrôle
\item
  Ajustement des effets visuels en fonction des tests
\end{itemize}

\hypertarget{gestion-de-projets}{%
\subsection{Gestion de projets}\label{gestion-de-projets}}

\begin{itemize}
\tightlist
\item
  Schématisation
\item
  Prototypage
\item
  Gestion de banques d'images
\item
  Optimisation des performances de l'application
\item
  Test de contrôle de qualité
\item
  Préréglages
\item
  Optimisation de la programmation et commentaires
\item
  Console de débogage
\item
  Exportation de projets
\item
  Formats de sauvegarde\\
\item
  Application autonome
\item
  Sauvegarde et archivage des médias
\end{itemize}

\hypertarget{historique}{%
\chapter{Historique du traitement vidéo}\label{historique}}

\hypertarget{uxe9volution-historique-du-traitement-viduxe9o-dans-les-diffuxe9rentes-formes-dart}{%
\section{Évolution historique du traitement vidéo dans les différentes formes d'art}\label{uxe9volution-historique-du-traitement-viduxe9o-dans-les-diffuxe9rentes-formes-dart}}

\hypertarget{performance}{%
\subsection{Performance}\label{performance}}

\hypertarget{installation}{%
\subsection{Installation}\label{installation}}

\hypertarget{uxe9volution-des-technologies-associuxe9es}{%
\subsection{Évolution des technologies associées}\label{uxe9volution-des-technologies-associuxe9es}}

\hypertarget{langages-et-moyens-expressifs-de-limage-en-mouvement}{%
\section{Langages et moyens expressifs de l'image en mouvement}\label{langages-et-moyens-expressifs-de-limage-en-mouvement}}

\hypertarget{lexique}{%
\chapter{Lexique technique et technologique}\label{lexique}}

\hypertarget{formats-de-fichiers}{%
\section{Formats de fichiers}\label{formats-de-fichiers}}

\hypertarget{encodage-des-viduxe9os}{%
\section{Encodage des vidéos}\label{encodage-des-viduxe9os}}

\hypertarget{captation-viduxe9o-en-temps-ruxe9el}{%
\section{Captation vidéo en temps réel}\label{captation-viduxe9o-en-temps-ruxe9el}}

\hypertarget{logiciels-de-traitement-viduxe9o-en-temps-ruxe9el-et-dinteractivituxe9}{%
\section{Logiciels de traitement vidéo en temps réel et d'interactivité}\label{logiciels-de-traitement-viduxe9o-en-temps-ruxe9el-et-dinteractivituxe9}}

\hypertarget{logiciels-de-programmation-nodale}{%
\section{Logiciels de programmation nodale}\label{logiciels-de-programmation-nodale}}

\hypertarget{notions-de-traitement-viduxe9o}{%
\section{Notions de traitement vidéo}\label{notions-de-traitement-viduxe9o}}

\hypertarget{pixels}{%
\subsection{Pixels}\label{pixels}}

\hypertarget{couleurs}{%
\subsection{Couleurs}\label{couleurs}}

\hypertarget{texture}{%
\subsection{Texture}\label{texture}}

\hypertarget{matrice}{%
\subsection{Matrice}\label{matrice}}

\hypertarget{muxe9moire-tampon}{%
\subsection{Mémoire tampon}\label{muxe9moire-tampon}}

\hypertarget{alpha-channel}{%
\subsection{Alpha channel}\label{alpha-channel}}

\hypertarget{rendu-opengl}{%
\subsection{Rendu OpenGL}\label{rendu-opengl}}

\hypertarget{traiter}{%
\chapter{Traiter l'image en mouvement}\label{traiter}}

\hypertarget{usage-de-capture-viduxe9o-en-temps-ruxe9el}{%
\section{Usage de capture vidéo en temps réel}\label{usage-de-capture-viduxe9o-en-temps-ruxe9el}}

\hypertarget{effets-visuels-et-filtres-applicables-en-temps-ruxe9el-sur-des-matuxe9riaux-visuels}{%
\section{Effets visuels et filtres applicables en temps réel sur des matériaux visuels}\label{effets-visuels-et-filtres-applicables-en-temps-ruxe9el-sur-des-matuxe9riaux-visuels}}

\hypertarget{traitement-visuel-en-temps-ruxe9el-uxe0-laide-deffets-et-de-logiciels-de-programmation-multimuxe9dia-et-nodale}{%
\section{Traitement visuel en temps réel à l'aide d'effets et de logiciels de programmation multimédia et nodale}\label{traitement-visuel-en-temps-ruxe9el-uxe0-laide-deffets-et-de-logiciels-de-programmation-multimuxe9dia-et-nodale}}

\hypertarget{flot-de-donnuxe9es-entre-les-objets-du-logiciel}{%
\section{Flot de données entre les objets du logiciel}\label{flot-de-donnuxe9es-entre-les-objets-du-logiciel}}

\hypertarget{exploitation-des-fonctions-des-logiciels-de-traitement-viduxe9o-en-temps-ruxe9el}{%
\section{Exploitation des fonctions des logiciels de traitement vidéo en temps réel}\label{exploitation-des-fonctions-des-logiciels-de-traitement-viduxe9o-en-temps-ruxe9el}}

\hypertarget{utilisation-de-nuanciers-shaders}{%
\section{Utilisation de nuanciers (shaders)}\label{utilisation-de-nuanciers-shaders}}

\hypertarget{programmer}{%
\chapter{Programmer des effets visuels}\label{programmer}}

\hypertarget{programmation-de-compositions-visuelles-guxe9nuxe9ratives}{%
\section{Programmation de compositions visuelles génératives}\label{programmation-de-compositions-visuelles-guxe9nuxe9ratives}}

\hypertarget{ruxe9alisation-dun-uxe9chantillonneurmuxe9langeur-visuel}{%
\section{Réalisation d'un échantillonneur/mélangeur visuel}\label{ruxe9alisation-dun-uxe9chantillonneurmuxe9langeur-visuel}}

\hypertarget{programmation-pour-contruxf4ler-la-lecture-viduxe9o}{%
\section{Programmation pour contrôler la lecture vidéo,}\label{programmation-pour-contruxf4ler-la-lecture-viduxe9o}}

\hypertarget{montage-temps-ruxe9el}{%
\subsection{montage temps réel}\label{montage-temps-ruxe9el}}

\hypertarget{niveau-des-couleurs}{%
\subsection{niveau des couleurs}\label{niveau-des-couleurs}}

\hypertarget{alpha-channel-1}{%
\subsection{alpha channel}\label{alpha-channel-1}}

\hypertarget{programmation-nodale-pour-cruxe9er-des-effets-en-temps-ruxe9el}{%
\section{Programmation nodale pour créer des effets en temps réel}\label{programmation-nodale-pour-cruxe9er-des-effets-en-temps-ruxe9el}}

\hypertarget{position}{%
\subsection{position}\label{position}}

\hypertarget{rotation}{%
\subsection{rotation}\label{rotation}}

\hypertarget{dimensions}{%
\subsection{dimensions}\label{dimensions}}

\hypertarget{mixage-dimages}{%
\subsection{mixage d'images}\label{mixage-dimages}}

\hypertarget{incrustation}{%
\subsection{incrustation}\label{incrustation}}

\hypertarget{distorsion}{%
\subsection{distorsion}\label{distorsion}}

\hypertarget{duxe9lais}{%
\subsection{délais}\label{duxe9lais}}

\hypertarget{ruxe9troaction-feedback}{%
\subsection{rétroaction (feedback)}\label{ruxe9troaction-feedback}}

\hypertarget{modification-de-couleurs}{%
\subsection{modification de couleurs}\label{modification-de-couleurs}}

\hypertarget{chromakey}{%
\subsection{chromakey}\label{chromakey}}

\hypertarget{lumiuxe8re}{%
\subsection{lumière}\label{lumiuxe8re}}

\hypertarget{fumuxe9e}{%
\subsection{fumée}\label{fumuxe9e}}

\hypertarget{texture-1}{%
\subsection{texture}\label{texture-1}}

\hypertarget{nuanceurs-shaders-vertex-pixel-et-guxe9omuxe9trie}{%
\section{Nuanceurs (shaders)\,: vertex, pixel et géométrie}\label{nuanceurs-shaders-vertex-pixel-et-guxe9omuxe9trie}}

\hypertarget{interagir}{%
\chapter{Interactivité et images en mouvement}\label{interagir}}

\hypertarget{intuxe9gration-des-composantes-dans-une-production-interactive}{%
\subsection{Intégration des composantes dans une production interactive}\label{intuxe9gration-des-composantes-dans-une-production-interactive}}

\hypertarget{configuration-logicielle-et-matuxe9rielle-dune-production-interactive}{%
\subsection{Configuration logicielle et matérielle d'une production interactive}\label{configuration-logicielle-et-matuxe9rielle-dune-production-interactive}}

\hypertarget{conceptualisation-et-scuxe9narisation-dun-projet-visuel-interactif}{%
\subsection{Conceptualisation et scénarisation d'un projet visuel interactif}\label{conceptualisation-et-scuxe9narisation-dun-projet-visuel-interactif}}

\hypertarget{captation-de-mouvement-et-de-pruxe9sence}{%
\subsection{Captation de mouvement et de présence}\label{captation-de-mouvement-et-de-pruxe9sence}}

\hypertarget{programmation-de-la-captation-de-mouvement-et-de-pruxe9sence}{%
\subsection{Programmation de la captation de mouvement et de présence}\label{programmation-de-la-captation-de-mouvement-et-de-pruxe9sence}}

\hypertarget{utilisation-dinterfaces-de-contruxf4le-interactives}{%
\subsection{Utilisation d'interfaces de contrôle interactives}\label{utilisation-dinterfaces-de-contruxf4le-interactives}}

\hypertarget{utilisation-dosc-midi-dmx-ou-artnet-pour-interagir-avec-dautres-logiciels-et-interfaces-de-contruxf4le}{%
\subsection{Utilisation d'OSC, MIDI, DMX ou ArtNet pour interagir avec d'autres logiciels et interfaces de contrôle}\label{utilisation-dosc-midi-dmx-ou-artnet-pour-interagir-avec-dautres-logiciels-et-interfaces-de-contruxf4le}}

\hypertarget{ajustement-des-effets-visuels-en-fonction-des-tests}{%
\subsection{Ajustement des effets visuels en fonction des tests}\label{ajustement-des-effets-visuels-en-fonction-des-tests}}

\hypertarget{duxe9ploiement-de-projet-viduxe9o-interactif}{%
\chapter{Déploiement de projet vidéo interactif}\label{duxe9ploiement-de-projet-viduxe9o-interactif}}

\hypertarget{schuxe9matisation}{%
\section{Schématisation}\label{schuxe9matisation}}

\hypertarget{prototypage}{%
\section{Prototypage}\label{prototypage}}

\hypertarget{gestion-de-banques-dimages}{%
\section{Gestion de banques d'images}\label{gestion-de-banques-dimages}}

\hypertarget{optimisation-des-performances-de-lapplication}{%
\section{Optimisation des performances de l'application}\label{optimisation-des-performances-de-lapplication}}

\hypertarget{test-de-contruxf4le-de-qualituxe9}{%
\section{Test de contrôle de qualité}\label{test-de-contruxf4le-de-qualituxe9}}

\hypertarget{pruxe9ruxe9glages}{%
\section{Préréglages}\label{pruxe9ruxe9glages}}

\hypertarget{optimisation-de-la-programmation-et-commentaires}{%
\section{Optimisation de la programmation et commentaires}\label{optimisation-de-la-programmation-et-commentaires}}

\hypertarget{console-de-duxe9bogage}{%
\section{Console de débogage}\label{console-de-duxe9bogage}}

\hypertarget{exportation-de-projets}{%
\section{Exportation de projets}\label{exportation-de-projets}}

\hypertarget{formats-de-sauvegarde}{%
\section{Formats de sauvegarde}\label{formats-de-sauvegarde}}

\hypertarget{application-autonome}{%
\section{Application autonome}\label{application-autonome}}

\hypertarget{sauvegarde-et-archivage-des-muxe9dias}{%
\section{Sauvegarde et archivage des médias}\label{sauvegarde-et-archivage-des-muxe9dias}}

\hypertarget{exercices}{%
\chapter{Exercices}\label{exercices}}

\hypertarget{premier}{%
\section{Premier}\label{premier}}

\hypertarget{deuxiuxe8me}{%
\section{Deuxième}\label{deuxiuxe8me}}

\hypertarget{troisiuxe8me}{%
\section{Troisième}\label{troisiuxe8me}}

  \bibliography{book.bib,packages.bib}

\end{document}
